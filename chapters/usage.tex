% vim: spell spelllang=cs
\chapter{Použití šablony}

Je to tak triviální, až se z~toho člověku obklopenému všemožnými bloated
nesmysly chce brečet.

\begin{enumerate}
    \item Stáhněte si kostru do adresáře na závěrečnou práci.
    \item Otevřete \texttt{thesis.tex} a~vyhledávejte text \texttt{XXX}, jímž
    jsou označeny části, které určitě budete chtít upravit.
    \begin{itemize}
        \item Surprise twist! Části \texttt{XXX} vás mohou poslat do jiných
        souborů, v~nichž se nachází jiné části \texttt{XXX}, které budete chtít
        upravit! Říká se tomu rekurse a~je to princip a~nástroj, bez jehož
        pochopení se do psaní své nezáživné práce možná radši ani nepouštějte.
        \item Chcete změnit něco, co nemá \texttt{XXX}? Nelíbí se vám můj výběr
        barev? Toužíte po jiných okrajích? No tak to změňte, ta kostra se z~toho
        (asi) neposere.\footnote{ono teda co v~\LaTeX u se sem tam „samo“ neposere}
    \end{itemize}
    \item Obsah si narvěte, kam chcete; já navrhuji podadresář
    \texttt{chapters}. Jistě pochopíte, kam ho v~hlavním souboru zařadit.
    Přiložený makefile s~tím trochu počítá, ale přiložený makefile není svatý
    a~upravit ho zvládnete i~se znalostmi z~PB152cv nebo kde se to teď učí.
    \begin{itemize}
        \item Při vkládání literatury chcete do \texttt{bibliography.bib}
        yoinknout to, co se se jmenuje \emph{záznam BibTeX} nebo nějak podobně.
    \end{itemize}
    \item Jestli máte `latexmk`, tak už stačí `make` nebo (pro živé přesazování)
    `make pvc`.
    \item Před odevzdáním povypínejte pracovní věci (hlavně režim
    \texttt{draft}).
\end{enumerate}

Odtud už to zvládnete sami, že?

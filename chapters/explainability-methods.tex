\chapter{Current Methods}

\todo{
    Promyslet, jestli vubec uvadet ty z dnesniho pohledu uz archaicke metody (WachterCF, LIME-C, SHAP-C),
    resp. jestli je nejak nespojit do mensi kapitoly.

    Tady bude prehled metod, ktere se daji vyuzit na counterfactuals.
    Rad bych cerpal z toho, co jsem sepsal do youtracku.
    \url{https://youtrack.rationai.cloud.e-infra.cz/articles/XAI-A-10/Research}
    Tu taxonomii si beru z toho prvniho clanku. Zaroven rozebrat,
    jaky je rozdil mezi intristic explainability a post-hoc explainability,
    s tim ze se budeme soustredit jen na post-hoc.
}

\section{Instance-Centric}

\todo{
    WachterCF, jakozto prvni vic uvadena post-hoc metoda na generovani
    counterfactuals.
}

\section{Regression-Centric}

\todo{
    LIME-C, jakozto metoda zalozena na LIME, coz je dost obecne znamy algoritmus.
    Uprimne nevim, jestli se mi timhle chce travit cas. Realne to asi nikdy moc nebylo ve hre,
    tohle si vybrat. Stejne jako SHAP-C. Zatim zminuju jen pro uplnost.
}

\section{Game Theory Centric}

\todo{
    SHAP-C
}

\section{Probability-Centric}

\todo{
    Tohle je to, z ceho jsme realne vybirali.
}

\subsection{Generative Adversarial Networks}

\todo{
    Vybrat z clanku pristupy k generovani skrz GAN.
    Generator vs Discriminator, tezke na trenovani, ale docela dobre
    a rychle nagenerovane vysledky. Muze mit problem s diversitou.
}

\subsection{Variational Autoencoders}

\todo{
    Vybrat z clanku pristupy k generovani skrz VAE.
    Encoder + Decoder + regularizace latent space, lehke na natrenovani,
    rychle vysledky, ktere ale jsou casto blurry. Latentni prostor.
}

\subsection{Diffusion Models}

\todo{
    Vybrat z clanku pristupy k generovani skrz diffusion.
    Postupne oddelavani sumu, pomaly trenink a inference,
    ale dobre diverzni vysledky. Nema latentni prostor.
    Vybrali jsme si, protoze je to ma dost dobre vysledky
    a latentni prostor se da pridat, architekturou DAE.
}
% vim: spell spelllang=cs
\chapter{Úvod}

Protože trend rozežraných webových stránek už svoji lehce vulgární odpověď
má~\cite{web:mf}, rozhodl jsem se v~podobném duchu ukázat, že \LaTeX ová kostra
závěrečné práce může existovat v~analogicky jednoduché podobě.

,,Fakulta``\footnote{což je fskutečnosti entita, která neexistuje, až by si
v~ní člověk objednal nápoj (a~zaplatil za něj osm korun)} od pradávna
shilluje svým studentům šablonu \texttt{fithesis}, která sice už podle názvu
zní, že je \emph{the perfect fit for the job} a~slibuje, že všechny náležitosti
bakalářky a~diplomky vyřeší za vás, ale je neestetická, nevzhledná, hnusná;
škaredá i~šeredná; ohyzdná, ohavná, odporná i~otřesná a~nádavkem ještě k~zblití.

Prozradím vám teď tajemství: na sazbu své sračky, za niž chcete dostat titul,
nepotřebujete žádnou zázračnou externí šablonu; všechno co potřebujete, jste si
nainstalovali se základním \LaTeX ovým balíkem. (Pomiňme teď Overleaf; jestli
nutně potřebujete cpát každé usrání do obláčku, určitě to někdo časem překlopí
i~tam.)

Možná jste nabyli dojmu, že ten fithesistní nevkus vás aspoň zbaví některých
starostí s~\TeX em, a~že je to tedy skvělá volba, pokud jste studovali nějaký
odpadní obor, během něhož jste na tento osmdesátkový klenot nemuseli sahat.
Špatně! Je to úplně stejný krám jako zbytek \LaTeX ového světa, akorát se k~němu
ještě trochu hůř kachní\footnote{basic bitches si můžou doplnit ,,googlí``}
rovnáky na ohýbáky, protože nevíte, které narovnání nedopatřením ohnete uvnitř
šablony, o~níž zbytek světa jaktěživ neslyšel (šťastné to lidstvo).

Zahoďte tedy obavy, vy\TeX at bakalářku zvládli větší lempli než vy a~příkazy na
formátování obsahu jsou úplně stejné pro fithesi, jako tady, v~relativně čistém
\LaTeX ovém \texttt{book}. Na ten zbytek máte tuhle kostru, která ale nic
divnějšího nepoužívá a~hlavně vám všechno odhaluje a~umožňuje změnit přímo, bez
nějakých pomocných maker nebo sahání kdoví kam.

Nechci tu tvrdit, že \texttt{book} je kdovíjaký typografický skvost, na druhou
stranu na předčení úrovně fithese stačí žalostně málo. Jestli vám na
estetičnosti sazby fakt záleží (což v~době elektronicky odevzdávaných prací
popravdě ztrácí na významu) a~z~\LaTeX u strach nemáte, \emph{prosím}, všemi
deseti sáhněte po něčem hezčím (osobně mám rád šablonu
\texttt{mimosis}\footnote{\url{https://github.com/Pseudomanifold/latex-mimosis}}).
Tenhle počin je odpovědí na argument „já s~tím lateksem neumím a~ta fithesis už
má všechno nastavené, tak to bude nejlepší volba“. Není. Jestli máte celou
bakalářku tak na háku, že byste použili fithesi, použijte radši tohle -- bude to
mít úplně stejný „žádný velký sraní“ vajb, ale ve výsledku to bude  vypadat líp.

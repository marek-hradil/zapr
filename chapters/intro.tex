% vim: spell spelllang=en
\chapter{Introduction}

\todo{
    Tady bude nějaký obecný úvod o tom, proč je potřeba explainabity,
    tj. bez toho abychom rozuměli DNN je nemůžeme nasadit do safety-critical odvětví,
    protože nejsme schopni dát bezpečnostní záruky.
    Jedno z těch odvětví je histopatologie, kde by pokažená diagnostika pacienta mohla být
    velký špatný.
}

\section{Digital Histopathology}

\todo{
    Tady proč dává smysl histopatologii digitalizovat a zavést do ní AI,
    tj. lidi mají vysoký error rate, těžko se na velkých snímicích něco hledá,
    velký rozdíl mezi juniorními a seniorními patology, zároveň AI systémy, které už do odvětví pronikly
    snižují error rate \citneeded. Potom taky trochu o tom, jaké challenges zavádění AI
    do medicíny má, že je důležité aby patologové byli schopni porozumět rozhodnutím a důvěřovali systému.
}

\section{Explainability}

\todo{
    Vysvětlitelnosti a interpretovatelnosti AI se zabírá obor explainability,
    které sdružuje metody, které se snaží vysvětlit rozhodnutí AI systémů.
    Dnes hlavní focus na hluboké neuronové sítě, kvůli probíhajícímu rozmachu.
    Nějaký menší odstavec s přehledem co se všechno dá v explainability dělat,
    co se zkoušelo a že ve výsledku jde nejvíc o to, jak to dává konečným uživatelům
    smysl a jak jsou schopni na základě výsledku interpretovat, co se tam děje.
}

\section{Counterfactuals}

\todo{
    Jedna z metod je generování protipříkladů. Popíše se co to vlastně je,
    jaký by měl být ideální end-result. Pak zmínit, že vysvětlitelnost z protipříkladů gut,
    protože se zjistilo, že useři tomu rozumí \citneeded, tj. dává smysl to víc zkoumat a rozvíjet.
}

\section{Application to Histopathology}

\todo{
    Zamyslet se nad lepsim nazvem.
    Něco o tom co jsou WSI, co jsou tiles, jaké máme v RationAI již fungující modely
    a tím pádem, že ty counterfactuals budeme chtít generovat na tilech.
}